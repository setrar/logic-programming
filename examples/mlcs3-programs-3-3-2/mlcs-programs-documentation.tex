\documentclass[11pt]{article}

\usepackage{mathptmx}
\usepackage{url}

\newcommand*{\p}[1]{\textup{\texttt{#1}}}

\textwidth=15cm
\textheight=22.5cm
\topmargin=0pt
\headheight=0pt
\oddsidemargin=1cm
\headsep=0pt
\renewcommand{\baselinestretch}{1.1}
\setlength{\parskip}{0.20\baselineskip plus 1pt minus 1pt}
\parindent=0pt

\author{\bfseries Mordechai Ben-Ari\\\mbox{}\\\url{http: //www.weizmann.ac.il/sci-tea/benari/}}
\title{\bfseries Mathematical Logic for Computer Science\\\mbox{}\\
\bfseries\large (Third Edition)\\\mbox{}\\
\bfseries\Large Prolog Programs}

%\date{}
\begin{document}
\maketitle
\thispagestyle{empty}

\bigskip

\begin{center}
Copyright \copyright{} 2000--18 by Mordechai (Moti) Ben-Ari.
\end{center}
This work is licensed under the Creative Commons Attribution-ShareAlike 3.0
License. To view a copy of this license, visit
\url{http://creativecommons.org/licenses/by-sa/3.0/}; or, (b) send a letter
to Creative Commons, 543 Howard Street, 5th Floor, San Francisco,
California, 94105, USA.

\vfill
 
\begin{center}
The following copyright notice applies to the programs described in this
document:\mbox{}\\\mbox{}\\
Copyright 2000--18 by M. Ben-Ari.
\end{center}

This program is free software; you can redistribute it and/or
modify it under the terms of the GNU General Public License
as published by the Free Software Foundation; either version 2
of the License, or (at your option) any later version.
This program is distributed in the hope that it will be useful
but WITHOUT ANY WARRANTY; without even the implied warranty of
MERCHANTABILITY or FITNESS FOR A PARTICULAR PURPOSE.
See the GNU General Public License for more details.
You should have received a copy of the GNU General Public License
along with this program; if not, write to the Free Software
Foundation, Inc., 59 Temple Place - Suite 330, Boston, MA
02111-1307, USA.

\newpage

\section{Source archive}\label{s.archive}

The programs have been tested with SWI-Prolog Version 6.0.0
\url{http://www.swi-prolog.org/}.

The source files use the extension \p{pro} instead of
\p{pl} to avoid conflict with program in Perl.

The archive is a zip file structured into directories (see Appendix A
for the list of files):

\begin{itemize}
\item \p{common} - modules and files used by other programs
\item \p{prop}   - propositional logic
\item \p{fol}    - first-order logic
\item \p{tl}     - temporal logic
\end{itemize}

\textbf{The SAT solver programs have been moved to a separate project called LearnSAT.}

\section{Common modules}\label{s.common}

The operators in the logics are represented within the Prolog programs
using the following operators defined in the file \p{ops}:

\begin{verbatim}
:- op(650, xfy, xor).        /* exclusive or */
:- op(650, xfy, eqv).        /* equivalence  */
:- op(650, xfy, nor).        /* nor          */
:- op(650, xfy, nand).       /* nand         */
:- op(640, xfy, imp).        /* implication  */
:- op(630, xfy, or).         /* disjunction  */
:- op(620, xfy, and).        /* conjunction  */
:- op(610, fy,  neg).        /* negation     */
:- op(600, xfy, eq).         /* equality     */
:- op(610, fy,  always).     /* always       */
:- op(610, fy,  eventually). /* eventually   */
:- op(610, fy,  next).       /* next         */
\end{verbatim}

Formulas written with this notation are not easy to read or write.
Instead, the following operators are used to for input and output:

\begin{verbatim}
:- op(650, xfy,+).          /* exclusive or */
:- op(650, xfy,<->).        /* equivalence  */
:- op(640, xfy,-->).        /* implication  */
:- op(630, xfy,v).          /* disjunction  */
:- op(620, xfy,^).          /* conjunction  */
:- op(610, fy, ~).          /* negation     */
:- op(610, fy, #).          /* always       */
:- op(610, fy, <>).         /* eventually   */
:- op(610, fy, @).          /* next         */
\end{verbatim}

Module \p{intext} contains predicates that translate between the
notations.

Other modules in the directory \p{common} are \p{defs} which contain the
semantic definitions of the Boolean operators and \p{io} which performs
input and output of the various logical formalisms. 

The predicate \p{write\_latex} writes a formula in \LaTeX{}.

Here is a formula in \LaTeX{} followed by its external and internal representations:
\begin{itemize}
\item $(p \oplus q)  \leftrightarrow (\neg (p \rightarrow q) \vee \neg (q\rightarrow p) )$.
\item \verb+(p xor q)  eqv (neg (p imp q) or neg (q imp p))+
\item \verb=(p + q) <-> (~(p -> q) v ~(q -> p))=
\end{itemize}

\section{Propositional logic}

\subsection{Truth tables}\label{s.tt}

The predicate \p{tt(Fml, V, TV)} returns the truth value \p{TV} of
formula \p{Fml} under the assignment \p{V}. The assignment is a list of
pairs \p{(A,TV)}, where \p{A} is an atom and \p{TV} is \p{t} or \p{f},
for example, \p{[(p,f),(q,t)]}. \p{tt} recurses on the structure of the
formula. For atoms, it returns the truth value by lookup in the list;
for negations, \p{neg} is called to negate the value; for formulas with
a binary operator, \p{opr} is called to compute the truth value from the
truth values of the subformulas.

\p{create\_tt(Fml)} prints the truth table for \p{Fml}.
\begin{verbatim}
create_tt(Fml) :-
  to_internal(Fml, IFml),
  get_atoms(IFml, Atoms),
  write_tt_title(IFml, Atoms),
  generate(Atoms, V),
  tt(IFml, V, TV),
  write_tt_line(IFml, V, TV),
  fail.
create_tt(_).
\end{verbatim}

\p{get\_atoms(Fml,Atoms)} returns a sorted list of the atoms occurring
in \p{Fml}. The assignments for this set of atoms are generated by
\p{generate(Atoms,V)}. As each assignment is generated, \p{tt(Fml, V,
TV)} is called, the value of \p{TV} is printed and then the predicate
\p{fail} causes backtracking into \p{generate} in order to print the
entire truth table.



\subsection{Semantic tableaux}\label{s.tabprop}

A tableau will be represented by a predicate \p{t(Fmls, Left, Right)},
where \p{Fmls} is a list of the formulas labeling the root of the
tableau, and \p{Left} and \p{Right} are the subtrees of the root which
recursively contain terms on the same predicate. \p{Right} is ignored
for an $\alpha$-rule. Here is the term for the tableau for $p \wedge
(\neg q \vee \neg p)$.

\begin{verbatim}
t([p and (neg q or neg p)],
  t([p, neg q or neg p],
    t([p,neg q],open,empty),
    t([p,neg p],closed,empty)
  ),
  empty
).
\end{verbatim}

The tableau for a formula \p{Fml} is created by starting with
\p{t([Fml],\_,\_)} and then extending the tableau by instantiating the
logical variables for the subtrees.

\begin{verbatim}
create_tableau(Fml, Tab) :-
  Tab = t([Fml], _, _), 
  extend_tableau(Tab).
\end{verbatim}

The predicate \p{extend\_tableau} performs one step of the tableau
construction. First, it checks for a pair of contradictory formulas (an
optimization, we don't wait until there are only literals) in \p{Fmls},
and then it checks if \p{Fmls} contains only literals. Only then does it
perform an alpha or a beta rule, with alpha rules given precedence. To
check if a branch is open, check if all elements of the label are
literals. To check if a branch is closed, a formula is chosen from the
list of formulas labeling the node, and then a search is made for its
complement. Backtracking will ensure that all possibilities are tried.

\begin{verbatim}
check_closed(Fmls) :-
  member(F, Fmls), member(neg F, Fmls).
\end{verbatim}

To perform an $\alpha$- or $\beta$-rule, we nondeterministically select
a formula, pattern-match it against the database of rules, delete the
formula from the node and add the subformulas. The rule for double
negation is implemented separately.



\subsection{Proof checker}\label{s.checkprop}

Just as we wrote a program to \emph{generate} a semantic tableau from a
formula, it would be nice if we could write a program to generate a
proof of a formula in $\mathcal{H}$. However, this is far from
straightforward as quite a lot of ingenuity goes into producing a
concise proof. In this section we present a \emph{proof checker}
for $\mathcal{H}$: a program which receives a list of formulas and their
assumptions as its input, and checks if the list is a correct proof. It
checks that each element of the list is either an axiom or assumption,
or follows from previous elements by \emph{MP} or deduction. The program
writes out the justification of each element in the list.
 
The axioms are facts with the axiom number as an additional argument.

The data structure used is a list whose elements are of the form
\p{deduce(A,F)}, where \p{A} is a list of formulas that is the current
set of assumptions, and \p{F} is the formula that has been proved. The
predicate \p{proof} has two additional arguments, a line number used on
output and a list of the formulas proved so far.

\begin{verbatim}
proof(List) :- proof(List, 0, []).
\end{verbatim}

Checking an axiom or an assumption is trivial and involves just checking
the database of axioms or the list of assumptions.

To check if \p{A} can be justified by \emph{MP}, the predicate \p{nth1}
nondeterministically searches \p{SoFar} for a formula of the form \p{B
imp A} and then for the formula \p{B}. The head of List is the Line'th
line in the proof, so the N'th element of the list is the Line-N'th
line.

\begin{verbatim}
proof([Fml | Tail], Line, SoFar) :- 
  Line1 is Line + 1,
  Fml = deduce(_, A),
  nth1(L1, SoFar, deduce(_, B imp A)), 
  nth1(L2, SoFar, deduce(_, B)),
  MP1 is Line1 - L1,
  MP2 is Line1 - L2,
  write_proof_line(Line1, Fml, ['MP ', MP1, ',', MP2]),
  proof(Tail, Line1, [Fml | SoFar]).
\end{verbatim}

A formula can be justified by the deduction rule if it is an implication
\p{A imp B}. Nondeterministically choose a formula from \p{SoFar} that
has \p{B} as its formula, and check that \p{A} is in its list of
assumptions. The formula \p{A} is deleted from \p{Assump}, the list of
assumptions of \p{A imp B}.



\subsection{Conversion to CNF}\label{s.cnf}

There are two programs: \p{cnfprop} for propositional logic and
\p{cnffol} that contains the additions for first-order logic.

The program for conversion to CNF performs three steps one after
another: elimination of operators other than negation, disjunction and
conjunction, reducing the scope of negations using De Morgan's laws and
double negation, and finally distribution of disjunction over
conjunction.

Elimination of operators is done by a recursive traversal of the
formation tree. The first three clauses eliminate \p{imp}, \p{eqv} and
\p{neqv}. The next four clauses simply traverse the tree for the other
operators.

The application of De Morgan's laws is similar. Two clauses apply the
laws for negations of conjunction and disjunction and the next two
clauses traverse the formula for non-negated formulas. The final two
clauses eliminate double negation and terminate the recursion at a
literal.

Distribution of disjunction over conjunction is more tricky, because one
step of the distribution may produce additional structures that must be
handled. For example, one step of distribution applied to $p \vee
((q\wedge r) \wedge r)$ gives $(p \vee (q\wedge r)) \wedge (p \vee r)$
and the distribution rule called recursively not just on the subformulas
but also on the new formula that results.

The predicate \p{cnf\_to\_clausal} transforms from formulas as terms
with operators to formulas as sets of sets. Both lists, clauses and
sets of clauses, are converted are converted to sets to remove
duplicates, and the literals in the clauses are sorted so that duplicate
clauses can be identified.

\subsection{Resolution}\label{s.resprop}

To perform resolution we first convert a CNF formula to clausal form.
Trivial clauses like $\{p,r,\bar{p}\}$ are discarded. First, a check is
made if the set of clauses is empty (the formula is valid) or if the set
contains the empty clause (the formula is unsatisfiable). Resolution is
performed by nondeterministically selecting two clauses in the set,
creating their resolvent, adding it to the set of clauses and
recursively calling the predicate. The predicate will fail and backtrack
in three cases: there are no clashing literals, the resolvent is
trivial, the resolvent already exists in the set.

\begin{verbatim}
resolve(S) :-
  member(C1, S),               % choose two clauses
  member(C2, S),               
  clashing(C1, L1, C2, L2),    % check that they clash
  delete(C1, L1, C1P),         % delete the clashing literals
  delete(C2, L2, C2P),         
  union(C1P, C2P, C),          % new clause is their union
  \+ clashing(C, _, C, _),     % don't add trivial clauses
  \+ member(C, S),             % don't add an existing clause
  write_clauses(S), nl,        
  resolve([C | S]).            % add the resolvent to the set
\end{verbatim}


\subsection{Binary decision diagrams}\label{s.bdd}

Atoms are represented by integers: \p{N} stands for the atom $p_{N}$.
BDDs are represented by the predicate \p{bdd(N,False,True)}, where \p{N}
is the atom labeling the root, \p{False} is the sub-BDD when \p{N} is
assigned $F$ and \p{True} is the sub-BDD when \p{N} is assigned $T$.

The module \p{bddwrite} contains predicates for formatting a BDD.

\subsubsection{Reduce}

Calling the predicate \p{reduce(B, BR)} with a BDD \p{B} returns the
reduced BDD in \p{BR}. It does a recursive traversal of the BDD. A cache
of reduced BDDs is maintained as a dynamic database, so that it is easy
to check if a BDD already exists as required by the second type of
reduction. \p{retractall} should be called before executing \p{reduce}
in order to initialize the database.

The first clause checks if the current BDD is in the cache; if so,
it is returned.

\begin{verbatim}
reduce(B, B) :- B, !.
\end{verbatim}

Next we check if the BDD is a leaf; if so, it is placed into the cache
and returned. The third clause recurses on the sub-BDDs, but before
returning it calls \p{remove} to perform the first type of reduction. If
the two edges from this node \p{N} point to the same subBDD, \p{N} must
be removed and one copy of the subBDD returned instead. Otherwise, the
new BDD formed by \p{N} and the subBDDs is returned after storing in the
cache.

\begin{verbatim}
remove(bdd(_, SubBDD, SubBDD), SubBDD) :- !.
remove(B, B) :- assert(B).
\end{verbatim}

\subsubsection{Apply}

\p{apply(B1, Opr, B2, B} applies the operator \p{Opr} to
the BDDs \p{B1} and \p{B2} and returns the result in \p{B}. A cache is
used for optimization: the predicate \p{bdd\_pair(B1, B2, B)} is
asserted if applying the operator to the pair of of BDDs \p{B1} and
\p{B2} and returns the result \p{B}. An additional optimization is to
integrate \p{reduce} with \p{apply} instead of first creating an
unreduced BDD.

\begin{verbatim}
apply1(B1, _, B2, Result) :-
  bdd_pair(B1, B2, Result), !.
apply1(B1, Opr, B2, Result1) :-
  create(B1, Opr, B2, Result),
  check_reduced(Result, Result1),
  assert(bdd_pair(B1, B2, Result1)).
\end{verbatim}

The algorithm requires a simultaneous recursive traversal of two BDDs.
The base case is if both BDDs are leaves; in this case, simply apply the
operator to the values in the leaves. If the same atom is at the root of
both BDDs, a simultaneous recursion is done and the resultant BDD
constructed from the BDDs that are returned.

When one of the BDDs has an atom at the root and the other is a leaf, or
when the roots are labeled with different atoms, the first clause is
taken if the right-hand sub-BDD is a leaf or has a higher-numbered atom;
in this case, the sub-BDDs of the left-hand node are applied to the
\emph{entire} right-hand node \p{Node2}. The two cases can be treated
together, using the \p{;} operator which succeeds if either of its
operand does. The check that \p{N1} is not a leaf is not needed by the
algorithm; it just ensures that we don't try to evaluate \p{leaf<N2}
which is illegal.

\begin{verbatim}
create(bdd(N1, False1, True1), Opr,     % True node is leaf or smaller
       Node2,
       bdd(N1, FalseResult, TrueResult)) :-
  Node2 = bdd(N2, _, _),                   
  (N2 = leaf ;
   (N1 \= leaf, N1 < N2)), !,           % Check N1\=leaf before "<"
  apply1(False1,  Opr, Node2, FalseResult),
  apply1(True1, Opr, Node2, TrueResult).
\end{verbatim}

There is another, symmetrical, clause if the left-hand node is a leaf or
has a higher-numbered atom.

Another optimization is to check for a \emph{controlling operand}
for the operator if one of the BDDs is a leaf. A value is controlling if
the result of the operation does not depend on the other operand. $T$ is
controlling for $\vee$ and $F$ is controlling for $\wedge$.

The restriction and quantification operations are also implemented.

\section{First-order logic}

\subsection{Semantic tableaux}\label{s.tabfol}

The predicate \p{check\_closed} that checks if a node is closed must use
the operator for syntactical identity \p{==} to prevent unification of
atomic formulas.

\begin{verbatim}
check_closed(Fmls) :-
  member(F1, Fmls), member(neg F2, Fmls), F1 == F2.
\end{verbatim}

To implement the systematic search, the rules are ordered so that rules
for $\alpha$-, $\beta$- and $\delta$-formulas are performed before
attempting the rule for a $\gamma$-formula. The tableau predicate has an
extra argument \p{C} to hold the list of constants. This is updated
whenever the rule for a $\delta$-formula is used.

The rules for the $\alpha$- and $\beta$-formulas are straightforward.

For a $\delta$-formula, \p{gensym} generates a new constant symbol.
\p{instance(A1,A2,X,C)} returns in \p{A2} the instance of \p{A1} that
can be obtained by replacing the variable \p{X} by the constant \p{C}.

\begin{verbatim}
delta_rule(Fmls, [A2 | Fmls1], C) :-
  member(A, Fmls),
  delta(A, X, A1), !,
  gensym(a, C),
  instance(A1, A2, X, C),
  delete(Fmls, A, Fmls1).
\end{verbatim}

For a $\gamma$-formula, first we have to identify if there exists a
$\gamma$-formula in the set of formulas. Then, each constant is used to
instantiate the $\gamma$-formula. The list of formulas is re-ordered:
the instantiated formulas are placed at the head of the list so that
non-$\gamma$-rules can be used if possible and the $\gamma$-formula is
placed at the end of the list so that other $\gamma$-formulas will be
used.

\begin{verbatim}
gamma_rule(Fmls, Fmls4, C) :-
  member(A, Fmls),
  is_gamma(A), !,               % Check if gamma rule
  gamma_all(C, A, AList),       % Apply gamma for all constants C
  delete(Fmls, A, Fmls1),       % Re-order: gamma(c), Fmls, gamma
  append(Fmls1, [A], Fmls2),    %   so that substitutions are taken
  append(AList, Fmls2, Fmls3),  %   and universal fml is taken last
  list_to_set(Fmls3, Fmls4).    % Remove duplicates introduced by gamma
\end{verbatim}

\p{gamma\_all(C, A, AList)} applies the predicate \p{gamma} to \p{A}
with all of the constants in the list \p{C} and returns the list of
formulas in \p{AList}. \p{gamma} recognizes the $\gamma$-formulas and
returns instances using \p{instance(A, A1, X, C)} in module
\p{instance}, where $A1$ is obtained from $A$ by instantiating $X$ by
the constant $C$. To create an instance, the predicate \p{instance}
recursively traverses the formula until an atomic formula is reached;
then the substitution is performed. To implement substitution, the
operator \p{==} must be used to prevent unification instead of
substitution. \p{instance} is more complex than it needs to be here
because it performs other tasks for proof checking.



\subsection{Proof checker}\label{s.checkfol}

The proof checker for the Hilbert system in propositional logic is
extend to first-order logic by adding Axioms~4 and~5, and the
generalization rule. Axiom~5 requires that the quantified variable not
occur as a free variable in the antecedent.

\begin{verbatim}
axiom(all(X, A1) imp A2, 4) :-
  instance(A1, A2, X, _).
axiom(all(X, A imp B) imp (A imp all(X, B)), 5) :-
  \+ free_in(A, X).
\end{verbatim}

Here, the predicate \p{instance(A, A1, X, C)}traverses the formulas
\p{A} and \p{A1} \emph{together}; when an atomic formula is reached, the
arguments are compared to see if they are the same variablev or if one
is a constant and the other the variable \p{X}. To check if a variable
is free in a formula, simply traverse the formula and for every
quantifier, check that the variable is different from the quantified
variable.

An additional argument \p{Gens} is added to the predicate \p{proof} to
store a list of the constants to which Generalization has been applied.
When the deduction rule is used, two things must be checked: that the
new set of assumptions is the same as the previous one without the
formula \p{A} and the proviso that no constant of \p{A} appears in
\p{Gens}. To check the proviso, the list of constants is traversed and a
check is made that each one does not appear (free) in \p{A}.

\begin{verbatim}
proviso([], _).
proviso([C|Rest], A) :-
  \+ free_in(A,C),
  proviso(Rest,A).
\end{verbatim}




\subsection{Conversion to CNF}

For first-order logic, there are additional predicates to rename the
bound variables and then extract the quantifiers, which is easy to do
once the variables have been renamed.

\p{rename} works by traversing the formula, keeping a list of variable
substitutions. The call is \p{rename(A,List,List1,A1)}, where \p{A1} is
\p{A} after the variables have been renamed, and \p{List} and \p{List1}
are lists of pairs of variables. On the way down the recursive traverse,
\p{List} stores all the variables that have been encountered and the new
variable names. At the bottom, \p{List} is unified with the variable
\p{List1} and the substitutions are made on the way up the recursive
traverse. When a quantified variable is the same as one previously
encountered, \p{copy\_term} is used to create a new variable. This is a
predicate that makes a copy of its first argument with a fresh variable
and places it in the second argument.

\begin{verbatim}
rename(all(X, A), List, List1, all(Y, A1)) :-
  member_var((X, _), List), !,
  copy_term(X, Y),
  rename(A, [(X, Y) | List], List1, A1).
\end{verbatim}

When a quantified variable is encountered for the first time, an
identity substitution is created. The clauses for \p{ex} are similar and
the clauses for the Boolean operators are elementary. The clause
terminating the recursion performs the substitution on the atomic
formulas using \p{subst\_var}.

A simple transversal is not sufficient for extracting quantifiers. A
simple traversal of the formula $(p_{1} \vee \forall xq_{1}(x)) \wedge
(p_{2} \vee \forall yq_{2}(y))$ will give $\forall x(p_{1} \vee
q_{1}(x)) \wedge \forall y(p_{2} \vee q_{2}(y))$, but the extraction has
to be applied again to this formula. To do this, the result of a
traversal is checked to see if the formula has changed and if so the
traversal is done again.




\subsection{Skolemization}\label{s.skolem}

The call to \p{skolem(A,A1)} first transforms the formula $A$ into CNF
and then calls \p{skolem(A, ListA, ListE,A1)} to obtain \p{A1}, the
Skolemized version of \p{A}. \p{ListA} is the list of universally
quantified variables that have appeared so far (initially, the empty
list) and \p{ListE} is a list of pairs: the first element is an
existentially quantified variable and the second element is itself a
pair that contains the Skolem function and a list of its arguments that
are to be substituted for the existential variable. \p{gensym} is used
to create new Skolem function symbols. Here are the clauses for
quantified formulas.

\begin{verbatim}
skolem(all(X, A), ListA, ListE, all(X, B)) :- !,
  skolem(A, [X | ListA], ListE, B).

skolem(ex(X, A), ListA, ListE, B) :- !,
  gensym(f, F),
  Function =.. [F | ListA],    
  skolem(A, ListA, [(X, Function) | ListE], B).
\end{verbatim}

The clauses for Boolean formulas are elementary. When an atomic
proposition is encountered in the recursive traversal,
the operator \p{=..} (read \emph{univ}) is used to decompose the formula
into a predicate symbol and a list of variables. Then, \p{subst\_var} is
called to replace existentially quantified variables by the Skolem
functions, and \p{=..} is called again to recompose the formula.




\subsection{Unification}\label{s.unif}

To unify a pair of atomic propositions, check that the predicate
symbols are identical, create a set of equations from the arguments and
call \p{solve}.

\begin{verbatim}
unify(A1, A2, Subst) :-
  A1 =.. [Pred | Args1],
  A2 =.. [Pred | Args2],
  create_equations(Args1, Args2, Eq),
  solve(Eq, Subst).
\end{verbatim}

The predicate \p{solve} is called with a list of equations and returns a
list of substitutions written as equations $x=t$, where $x$ is a
variable and $t$ is a term. The list is traversed, attempting to apply
each of the rules to the current equation. It is convenient to maintain
the list in two parts, one to the left of the current equation and one
that includes the current equation as its head and the rest of the
equations as its tail. The \p{solve} predicate will have four
parameters: two for the equation list, a third for status information
and the fourth will return the solved set.

The status is passed down the recursive calls to \p{solve} and is used
to terminate the recursion as necessary, for example, if either rule~3
or rule~4 fails. The equation that caused the failure is returned
together with the failure indication for printing. Each rule is applied
in turn; if successful, it sets the status to \p{modified} as an
indication to the list traversal clauses described below.

\begin{verbatim}
solve(Head, [Current | Tail], _, Result) :-
  rule1(Current, Current1), !,
  solve(Head, [Current1 | Tail], modified, Result).

solve(Head, [Current | Tail], _, Result) :-
  rule2(Current), !,
  solve(Head, Tail, modified, Result).

solve(Head, [Current | Tail], _, Result) :-
  rule3(Current, NewList, Status), !,
  append(NewList, Tail, NewTail),
  solve(Head, NewTail, Status, Result).
\end{verbatim}

The set of equations returned by rule~3 replaces the current equation
and is appended in front of the remaining equations.

When a substitution is performed in rule~4, it must be performed on
\emph{all} the equations, including those that have already been
checked.

\begin{verbatim}
solve(Head, [Current | Tail], _, Result) :-
  append(Head, Tail, List),
  rule4(Current, List, NewList, Status), !,
  solve([Current], NewList, Status, Result).
\end{verbatim}

The next three clauses of \p{solve} traverse the list. If no equation
applies, the traverse goes to the next one. When the end of the list is
reached, another traversal is initiated if the previous one made any
modifications to the list.

In the rules, we must prevent confusion between the equality operator of
the term equation and the Prolog equality operator, so the former is
explicitly defined using the operator \p{eq}. Rules~1 and~2 are
straightforward. Rule~3 compares the outermost functors and fails if
they are not the same. Otherwise, it calls \p{new\_equations} to pair
the subterms. Rule~4 reports failure if the occurs-check fails.
Otherwise, it calls \p{subst\_list} to perform the substitutions. If
nothing is changed, the predicate fails, initiating traversal to the
next equation in the list.

\p{occur(X,T)} traverses the list and succeeds as soon as it finds an
occurrence of the variable \p{X} in the term \p{T}. \p{occur\_list} is
used to check the list of subterms.




\subsection{Resolution}\label{s.resfol}

A formula is first transformed into a set of clauses using \p{skolem}
and \p{skolem\_to\_clauses} and then resolution is performed.

The empty set of clauses is valid and a set containing the empty clause
is unsatisfiable. Otherwise, choose two \emph{different} clauses and
resolve. \p{copy\_term} standardizes apart. After resolving a check is
made that the clause is not trivial and that it is a new clause.
\begin{verbatim}
resolve(S)  :-
  member(C1, S),
  member(C2, S),
  C1 \== C2,
  copy_term(C2, C2_R),
  clashing(C1, L1, C2_R, L2, Subst),
  delete_lit(C1, L1, Subst, C1P),
  delete_lit(C2_R, L2, Subst, C2P),
  clause_union(C1P, C2P, Resolvent),
  \+ clashing(Resolvent, _, Resolvent, _, _),
  \+ member(Resolvent, S),
  resolve([Resolvent | S]).
\end{verbatim}

The predicate \p{clashing(C1, L1, C2, L2, Subst)} checks if the clauses
clash and if so it returns the clashing literals and the mgu that
unifies them. \p{delete\_lit(Clause, Literal, Subst, Result)} deletes
from \p{Clause} all the literals that are equal to \p{Literal} under the
substitution \p{Subst} and returns the \p{Result}. \p{clause\_union(C1,
C2, Result)} takes the union of the literals in the two clauses to form
the resolvent.

\section{Temporal logic}

\subsection{Semantic tableaux}\label{s.tabtl}

The decision predicate for
satisfiability is implemented in four stages:
\begin{itemize}
\item \p{extend\_tableau} performs the tableau
construction until it terminates.
\item \p{check\_tableau} decides if the tableau is opened,
closed or contains cycles.
\item \p{create\_states} constructs the state diagram
from the tableau.
\item \p{check\_fulfillment}
constructs the component graph and checks fulfillment.
\end{itemize}

Each node of the tableau contains five fields \p{t(Fmls, Left, Right, N,
Path)}: the list of formulas and the links to the left and right
children. A node number that is generated by \p{get\_num} when a new
node is created and the ancestor path. This is used to check if a new
state should be created or if a node should be connected to an ancestor.
$\alpha$- and $\beta$-nodes add themselves to the ancestor path by
appending the term \p{pt(Fmls,N)} to \p{Path}, where \p{Fmls} is the
label and \p{N} is the node number. The rule for a $\beta$-formula is:

\begin{verbatim}
extend_tableau(t(Fmls, Left,   Right, N, Path)) :-
  beta_rule(Fmls, Fmls1, Fmls2), !,
  get_num(N1),
  get_num(N2),
  Left  = t(Fmls1, _, _, N1, [pt(Fmls,N)|Path]),
  Right = t(Fmls2, _, _, N2, [pt(Fmls,N)|Path]),
  extend_tableau(Left),
  extend_tableau(Right).
\end{verbatim}

The predicate \p{next\_rule} is called to get the formulas in a node
created by the rule for a next formula, but before the node is
created, \p{search} is called to search the \p{Path} for a node with the
same set of formulas in its label. If successful, it returns the node
number \p{N}; this branch of the tableau is terminated and marked
\p{connect(N)}. Otherwise, a new node is created.

\begin{verbatim}
extend_tableau(t(Fmls, connect(N), empty, _, Path)) :-
  next_rule(Fmls, Fmls1),
  search(Path, Fmls1, N), !.

extend_tableau(t(Fmls, Left,   empty, N, Path)) :-
  next_rule(Fmls, Fmls1), !,
  get_num(N1),
  Left = t(Fmls1, _, _, N1, [pt(Fmls,N)|Path]),
  extend_tableau(Left).
\end{verbatim}

After the tableau construction terminates, \p{check\_tableau} is called.
It traverses the tableau from the root down to the leaves and then
returns back up, computing the status of the tableau: open, closed or
with a cycle that must be checked for fulfillment.

\p{create\_states} takes a tableau and constructs the structure: states,
state paths which are transitions, and state labels which are the union
of the labels on the state paths. It returns a list of terms
\p{st(Fmls,~N)}, where \p{Fmls} is the state label and \p{N} the node
number of the state, and a list of terms \p{tau(From,~To)}, where
\p{From} and \p{To} are the node numbers of states.

These lists are the input to \p{component\_graph}, which returns a list
of MSCCs (a MSCC is a list of its states) and a list of edges of the
form \p{e(From,To)}, where \p{From} and \p{To} are MSCCs. \p{fulfil1}
selects a MSCC \p{S} with no outgoing edges and calls \p{self-fulfil} to
check if \p{S} is self-fulfilling. If successful, it returns \p{ok(S)};
if not, it deletes \p{S} from the list of MSCCs, adds
\p{notok(S,Result)} to the list of results and calls itself recursively.
\p{Result} is the future formula that could not be fulfilled in \p{S}.
\p{self-fulfil} checks each future formula such as \p{<>F} to see if
\p{F} occurs in some state in the SCC.


\appendix
\section{List of files}\label{s.list}

For each program \p{p.pro}, there is a file \p{p-t.pro} which contains
test programs. The programs use the predicate \p{file\_search\_path} to
access the modules in the \p{common} directory.

\smallskip

\begin{tabular}{l@{\hspace{3em}}l}
Directory \p{common}&\\
\p{ops.pro}      & declaration of operators with precedence and
associativity.\\
\p{def.pro}      & correspondence between symbols and internal operators.\\
            & semantic definition of Boolean operators.\\
\p{intext.pro}   & conversion from external to internal format and conversely.\\
\p{io.pro}       & display predicates for all programs (except BDDs and DPLL).\\
\\
Directory \p{prop} &  (propositional logic)\\
\p{tt.pro}       & truth tables.\\
\p{cnfpro.pro}   & conversion of a formula to CNF\\
\p{tabl.pro}     & semantic tableaux.\\
\p{check.pro}    & Hilbert proof checker.\\
\p{resolv.pro}   & resolution.\\
\p{bdd.pro}      & BDD algorithms.\\
\p{bddwrite.pro} & display of BDDs.\\
\\
Directory \p{fol}  & (first-order logic)\\
\p{cnffol.pro}   & conversion of a formula to CNF\\
\p{tabl.pro}     & semantic tableaux.\\
\p{check.pro}    & Hilbert proof checker.\\
\p{skolem.pro}   & skolemize a formula.\\
\p{unify.pro}    & unification algorithm.\\
\p{resolv.pro}   & resolution.\\
\p{instance.pro} & create an instance by substitution.\\
\\
Directory \p{tl}  &   (temporal logic)\\
\p{tl.pro}       & semantic tableaux.\\
\end{tabular}


\end{document}
